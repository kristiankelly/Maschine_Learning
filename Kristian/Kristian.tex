\documentclass{article}
\usepackage{graphicx}
\usepackage{amsmath, amssymb, amsthm} 
\usepackage{lmodern}                   
\usepackage{microtype}  
\usepackage[backend=biber, style=apa]{biblatex}
\usepackage{hyperref}
\addbibresource{References.bib}


\begin{document}
\section{Konvertierung von Musik in Maschinenlesbare Daten}
Bevor wir mit der Klassifikation und Generierung von Musikstücken beginnen können, müssen wir diese in ein Format umwandeln, welches von Maschinen gelesen und verarbeitet werden kann. 
Hierfür verwenden wir das MIDI-Format (Musical Instrument Digital Interface). MIDI-Dateien enthalten Informationen über Noten, Lautstärke, Tempo und andere musikalische Parameter, jedoch keine Audiodaten selbst.
Siehe \parencite{midi_smf_wikibooks} für eine ausführliche Beschreibung des MIDI-Formats und seine Struktur.

Im Rahmen dieses Projekts sind wir nur an den Notenwerten interessiert, sprich die Tonhöhe. Dabei wird jede Note durch eine (ganze) Zahl zwischen 0 und 127 dargestellt, wobei 60 dem mittleren C entspricht.
Das bedeutet natürlich, dass wir keine Informationen über die Klangfarbe, das Instrument, die Lautstärke oder die Dauer (Rhythmus) der Noten haben. Insofern 
führen wir hier also eine Art Dimensionsreduktion durch und "projezieren" die Musikstücke auf die Tonhöhe der Noten. Das würde dann in etwa so aussehen:


\begin{figure}[h]
  \centering
  \includegraphics[width=1\textwidth]{Typisches_stück.png}
  \caption{Musikstück nach Reduktion auf Tonhöhe der Noten}
  \label{fig:Typisches_stueck}
\end{figure}

Man sieht hier noch eine weitere Diskrepanz: Die Noten sind nicht immer einzeln nacheinander gespielt, sondern teilweise auch gleichzeitig (Akkorde). Daher führen wir noch eine weitere
Vereinfachung durch und ordnen die Noten in einen Vektor ein. Dabei geht natürlich wieder Information verloren. Wie genau das Preprocessing durchgeführt wird,
ist für die beiden Aufgabenstellungen (Klassifikation und Generierung) unterschiedlich und wird in den jeweiligen Abschnitten beschrieben.

\section{Musikgenerierung mit Recurrent Neural Networks}
\subsection{Preprocessing}
Für die Musikgenerierung verwenden wir das Bach-Chorales Dataset, welches ca. 400 Choräle von Johann Sebastian Bach enthält. Diese wurden aus MIDI-Dateien extrahiert und in ein geeignetes Format umgewandelt, welches
sich an den vier Stimmen (Sopran, Alt, Tenor, Bass) orientiert. Jede Stimme wird dabei als eine Sequenz von Notenwerten (ganzen Zahlen) dargestellt, siehe das unten stehende Beispiel.

\begin{figure}[h]
  \centering
  \includegraphics[width=1\textwidth]{Struktur_des_Chorals.png}
  \caption{Repräsentation eines Chorals im Bach-Chorales Dataset}
  \label{fig:Struktur_des_Chorals}
\end{figure}
Wie bereits im ersten Abschnitt erwähnt, wollen wir die Notenwerte in einem Vektor darstellen, was wir in unserem Preporcessing dann auch tun. Somit wir jedes Stück nun als eine 
Abfolge von ganzen Zahlen in $[0,127]_{\mathbb{Z}}$ dargestellt. 

Nun stellt sich die Frage, wie wir diese Daten unserem RNN zuführen. Konkret müssen wir 3 Hyperparameter festlegen:
\begin{itemize}
    \item $N$: Wie viele Noten sollen vorhergesagt werden?
    \item $L$: Wie viele vorherige Noten sollen betrachtet werden, um die nächsten Noten vorherzusagen?
    \item $O$: Das Offset, also wie viele Noten wir bei der nächsten Vorhersage überspringen.
\end{itemize}
Um das genauer zu erklären: Sei $S = [n_1, n_2, \ldots, n_k]$ eine Sequenz von Notenwerten. Dann wäre unser erstes Input-Output-Paar für das Training:
\[ [n_1,\ldots,n_L] \rightarrow [n_{L+1},\ldots,n_{L+N}] \] und unser zweites Paar:
\[ [n_{1+O},\ldots,n_{L+O}] \rightarrow [n_{L+1+O},\ldots,n_{L+N+O}] \] und so weiter.
\subsection{Musikgenerierung}
Bevor wir unsere verschiedenen Versuche vorstellen, wollen wir kurz erklären, wie die Musikgenerierung mit dem trainierten Modell funktioniert.
Gegeben ein Anfangssequenz von Noten $[n_1,\ldots,n_L]$, können wir das Modell verwenden, um die nächsten $N$ Noten vorherzusagen. Dabei erweist es sich als nützlich,
nicht deterministisch zu arbeiten, sondern den Output der softmax Schicht als Wahrscheinlichkeitsverteilung zu interpretieren und daraus eine Note zu sampeln, also NICHT 
immer die Note mit der höchsten Wahrscheinlichkeit zu wählen.

Diese vorhergesagten Noten werden dann an das Ende der Sequenz angehängt,
und wir nehmen wieder die letzten $L$ Noten, um die nächsten $N$ Noten vorherzusagen. Diesen Prozess wiederholen wir, bis wir die gewünschte Länge des generierten Musikstücks erreicht haben.

Da es in Latex sehr schwierig ist, Audiodateien einzubetten, findet sich hier ein Link zu einem Kaggle Notebook,
welches alle generierten Musikstücke mit Audiodateien enthält: \url{https://www.kaggle.com/code/kristiankelly/audiodateien-fuer-ausarbeitung}

\subsection{Ein erster Versuch}
Gerade stehen wir ganz am Anfang, wir können also nur raten, wie wir $N,L,O$ wählen und welche Architektur wir verwenden sollen. Wir haben auf gut Glück $N = 31$ gesetzt (damit wir 
nicht immer auf einer Bass-Note enden). Für 
$L$ scheint 4 eine gute Wahl zu sein, da wir ja 4 Stimmen haben. Für $O$ haben wir uns für 1 entschieden, also immer eine Note weiter. Unsere Architektur ist ein einfaches RNN, welches
wie strukturell für den Rest des Projekts verwenden werden, siehe Abbildung \ref{fig:Einfaches_RNN_N=4,L=32,O=1}.
\begin{figure}[h]
  \centering
  \includegraphics[width=1\textwidth]{Model_N=4,L=32,O=1.png}
  \caption{Einfaches RNN mit $N=31,L=4,O=1$}
  \label{fig:Einfaches_RNN_N=4,L=32,O=1}
\end{figure}

Wir heben folgende Dinge hervor:
\begin{itemize}
    \item Wir verwenden eine Embedding-Schicht, um die Notenwerte in einen dichteren Vektorraum (in diesem Fall $\mathbb{R}^{64}$) abzubilden, was dem Netzwerk helfen soll, siehe z.B. \parencite{chollet2021deep} 
    Kapitel 14. Es ist nämlich bei Noten wie bei Sprachen so, dass Noten, deren Abstände sehr kurz sind (z.B. C und C\#) nicht unbedingt eine ähnliche Bedeutungen/Wirkung haben.
          Somit muss das Netzwerk diese Beziehungen selbst lernen.
    \item Wir verwenden zwei LSTM (Long Short-Term Memory) Zellen, welche sehr vereinfacht gesagt in der Lage sind, Langzeitabhängigkeiten in Sequenzen zu lernen, siehe z.B. \parencite{chollet2021deep} Kapitel 13 für eine ausführliche Beschreibung.
    \item Am Ende nehmen wir uns die letzten $N=4$ Ausgaben und leiten diese über eine Dense-Schicht mit Softmax-Aktivierung, um eine Wahrscheinlichkeitsverteilung über die möglichen Notenwerte zu erhalten. 
    Da die höchsten und tiefsten Notenwerte im Dataset nicht vorkommen, können wir auf 88 reduzierem, also die Tasten eines Klaviers.
\end{itemize}

Wir trainieren für 20 Epochen mit Early Stopping und erhalten eine val-accuracy von ca. $0.82$ und eine test-accuracy von ca. $0.80$. NICHT SCHLECHT! Ein generiertes Musikstück sieht 
in Abbildung \ref{fig:Generiertes_Stueck_N=32,L=4,O=1} (die ersten zwei Takte sind vorgegeben, der Rest wurde generiert).
\begin{figure}[h]
  \centering
  \hspace*{-1.5cm}
  \includegraphics[width=1.2\textwidth]{Musik_Modell_1.png}
  \caption{Generiertes Musikstück mit $N=32,L=4,O=1$}
  \label{fig:Generiertes_Stueck_N=32,L=4,O=1}
\end{figure}
Uns fallen zwei Dinge auf. Zum einen werden die Akkorde oft wiederholt, was jedoch zu erwarten war, da wir die Rythmus Komponente komplett ignorieren, weswegen eine ganze Note
(= 4 Viertelnoten) als 4 gleiche Noten generiert wird. Zum anderen sehen (oder hören wir viel mehr), dass es dem Modell gelungen ist, gewisse Kadenzen (= typische Abfolgen von Akkorden) zu lernen, 
wie etwa die authentische Kadenz in Takte 6-7. Siehe \cite{amon2005lexikon}, Kapitel über Harmonie und die Funktionalität in Dur und Moll.

Trotzdem  beinhaltet das generierte Stück viele unharmonische Übergänge, sprich Kadenzen, die Harmonietheoretisch "keinen Sinn" ergeben. (Wieder verweisen wir bzgl. der genauen Bedeutungen
dieser Aussage auf \cite{amon2005lexikon}). Ein Grund hierfür könnte sein, dass wir immer 4 Noten auf einmal vorhersagen und so das Modell vielleicht "zu viel zu schnell" macht 
und daher Fehler auftauchen. Daher probieren wir im folgenden $N=1$ aus.
\subsection{Zweiter Versuch mit $N=1$}
Wir behalten alles bei wie vorhin, nur dass wir jetzt $N=1$ wählen, also immer nur eine Note vorhersagen. Wir trainieren für 30 Epochen mit Early Stopping und erhalten eine
val-accuracy von ca. $0.85$ und eine test-accuracy von ca. $0.83$. BESSER! Ein generiertes Musikstück sieht man in Abbildung \ref{fig:Generiertes_Stueck_N=32,L=1,O=1} (wieder sind die ersten zwei Takte vorgegeben, der Rest wurde generiert).
\begin{figure}[h]
  \centering
  \hspace*{-1cm}
  \includegraphics[width=1.2\textwidth]{Musik_Modell_2.png}
  \caption{Generiertes Musikstück mit $N=32,L=1,O=1$}
  \label{fig:Generiertes_Stueck_N=32,L=1,O=1}
\end{figure}
Dieses Stück ist harmonisch deutlich besser gelungen (es klingt nichts "falsch"). Jedoch ist dieses auch nur das beste aus 10 Versuchen, welches wir generiert haben.
Die anderen hatten fast alle zumindest ein paar Stellen, die sehr dissonant klangen. Im Barock (die Epoche, in der Bach gelebt hat) waren solche unharmonischen Stellen eher unüblich und man 
hat sich an strikte harmonische Regeln gehalten, wie etwa den Kontrapunkt \cite{tittel1959neue}. Als nächsten Schritt wollen wir versuchen, diese Regeln irgendwie in das 
Modell zu integrieren, um konsistenter harmonische Stücke zu generieren.
\subsection{Eine neue Loss Funktion}
Bis jetzt haben wir immer die Sparse-Categorical-Crossentropy als Loss Funktion verwendet. Nun wollen wir dissonante Fehler mehr bestrafen als konsonante. Dafür brauchen
wir zuerst ein Grundwissen über Intervalle, die Grundbausteine der Harmonielehre. Ein Intervall ist der Abstand zwischen zwei Tönen, gemessen in Halbtonschritten. Man unterscheidet
zwischen perfekten, konsonanten und dissonanten Intervallen \cite{amon2005lexikon}. Eine Übersicht über die wichtigsten Intervalle ist in der Tabelle \ref{tab:intervale mit konsonanz} dargestellt.
\begin{table}[h]
\centering
\begin{tabular}{|c|c|c|c|}
\hline
\textbf{Intervall} & \textbf{Halbtöne} & \textbf{Beispiel (von C)} & \textbf{Konsonanz} \\
\hline
Prime & 0  & C - C   & Perfekte Konsonanz \\
kleine Sekunde & 1  & C - C\# & Dissonanz \\
große Sekunde & 2  & C - D   & Dissonanz \\
kleine Terz & 3  & C - D\# & Konsonanz \\
große Terz & 4  & C - E   & Konsonanz \\
reine Quarte & 5  & C - F   & Perfekte Konsonanz \\
verminderte Quinte / Tritonus & 6  & C - F\# & Dissonanz \\
reine Quinte & 7  & C - G   & Perfekte Konsonanz \\
kleine Sexte & 8  & C - G\# & Konsonanz \\
große Sexte & 9  & C - A   & Konsonanz \\
kleine Septime & 10 & C - A\# & Dissonanz \\
große Septime & 11 & C - B   & Dissonanz \\
Oktave & 12 & C - C   & Perfekte Konsonanz \\
\hline
\end{tabular}
\caption{Musikalische Intervalle von Prime bis Oktave mit Konsonanz/Dissonanz \cite{amon2005lexikon}}
\label{tab:intervale mit konsonanz}
\end{table}
Ist also $v^{(n)} \in \mathbb{R}^{88}$ unsere Vorhersage für eine richtige Note $n \in [0,88]_{\mathbb{Z}}$, wollen wir einen Vektor $s^{(n)} \in \mathbb{R}^{88}$ definieren, welcher die Einträge von $v$ je nach Intervall zum vorhergesagten Ton bestraft.
Wir definieren $s^{(n)}$ wie folgt:
\[
s^{(n)}_i = 
\begin{cases}
0, & |i - n| = 0 \mod 12 \text{ (Prime)}\\
w_1, & |i - n| = 5,7 \mod 12 \text{ (perfekte Konsonanz)}\\
w_2, & |i - n| = 3,4,8,9 \mod 12 \text{ (Konsonanz)}\\
w_3, & |i - n| = 1,2,6,10,11 \mod 12 \text{ (Dissonanz)}\\
\end{cases}\]

mit $0 <w_1 < w_2 < w_3$ positiven Gewichten. Unser neuer Loss für eine Vorhersage $v^{(n)}$ mit richtiger Note $n$ ist dann:
\[\text{Loss}(v^{(n)},n) = \text{SparseCategoricalCrossentropy}(v^{(n)},n) + v^{(n)} \cdot s^{(n)}\] wobei $\cdot$ das euklidische Skalarprodukt ist.
Wir bemerken, dass der zweite Term genau dann 0 ist, wenn die Note korrekt vorhergesagt wurde, und umso größer wird, je "dissonanter" die Vorhersage ist, also genau dass, was wir wollten.

Trainieren wir nun dasselbe Modell wie im vorherigen Abschnitt mit dieser neuen Loss Funktion, wobei wir $w_1 = 1, w_2 = 2, w_3 = 3$ setzen, erhalten wir nach 30 Epochen eine val-accuracy von ca. $0.85$ und eine test-accuracy von ca. $0.84$.
Wir haben uns also dahingehend nicht wirklich verbessert. Aber wie schauen die generierten Musikstücke aus? Nunja, wir haben unser Ziel erfüllt: Die generierten Stücke sind harmonisch perfekt. 
Leider bedeutet "perfekt" in diesem Fall, das die ganze Zeit nur der erste Akkord (die Tonika) gespielt wird. Das gibt uns zwar das objektiv beste harmonische Stück,
aber es ist natürlich unglaublich langweilig. 

Wir starten einen zweiten Versuch mit $w_1 = 0.01, w_2 = 0.1, w_3 = 1$, also wir bewegen uns auf einer logarithmischen Skala.
Die val-accuracy und die test-accuracy bleiben gleich, aber die generierten Stücke sind nun das, was wir uns erhofft haben: Harmonisch sinnvoll und abwechslungsreich (und das bei fast jeder Generierung). 
Ein Beispiel ist in Abbildung \ref{fig:Generiertes_Stueck_N=32,L=1,O=1_Loss} dargestellt.

\begin{figure}[h]
  \centering
  \hspace*{-1cm}
  \includegraphics[width=1.2\textwidth]{Stück_neue_loss_funktion.png}
  \caption{Generiertes Musikstück mit $N=32,L=1,O=1$ und neuer Loss Funktion}
  \label{fig:Generiertes_Stueck_N=32,L=1,O=1_Loss}
\end{figure}

\newpage
\printbibliography
\end{document}



